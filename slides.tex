%===========================================
% Modelo de Exemplo Arquivo Beamer LaTeX
% Objetivo: usar em minicurso de introdução ao LaTeX
% Autoras: Luana Rios Mikolayczyk, Nara Bobko
% Versão: out.2023
%===========================================
\documentclass[aspectratio=169]{beamer}
\usepackage[utf8]{inputenc}
\usepackage[brazil]{babel}
\usepackage{multimedia,amsmath,graphicx,color,multicol,fancyhdr,amssymb,amsfonts,amsthm,setspace}

% Para mudar o tema dos slides:
%\usetheme{default}
%\usetheme{AnnArbor}
%\usetheme{Antibes}
%\usetheme{Bergen}
%\usetheme{Berkeley}
% \usetheme{Berlin}
%\usetheme{Boadilla}
%\usetheme{CambridgeUS}
%\usetheme{Copenhagen}
%\usetheme{Darmstadt}
%\usetheme{Dresden}
%\usetheme{Frankfurt}
%\usetheme{Goettingen}
%\usetheme{Hannover}
%\usetheme{Ilmenau}
% \usetheme{JuanLesPins}
%\usetheme{Luebeck}
%\usetheme{Madrid}
%\usetheme{Malmoe}
%\usetheme{Marburg}
%\usetheme{Montpellier}
%\usetheme{PaloAlto}
%\usetheme{Pittsburgh}
%\usetheme{Rochester}
\usetheme{Singapore}
%\usetheme{Szeged}
%\usetheme{Warsaw}

% Para mudar o pacote de cores dos slides:
%\usecolortheme{albatross}
%\usecolortheme{beaver}
%\usecolortheme{beetle}
%\usecolortheme{crane}
\usecolortheme{dolphin}
% \usecolortheme{dove}
%\usecolortheme{fly}
%\usecolortheme{lily}
%\usecolortheme{monarca}
%\usecolortheme{seagull}
%\usecolortheme{seahorse}
%\usecolortheme{spruce}
%\usecolortheme{whale}
%\usecolortheme{wolverine}

% Informações que serão inseridas no slide da capa:
\title[Título curto]{Título completo da apresentação}
\author[Nome abreviado]{Seu nome completo}
\institute[UTFPR]{Universidade Tecnológica Federal do Paraná - UTFPR \\ Campus Curitiba}
\date{\today}
%Comando para colocar um logotipo em todos os slides. Dependendo do tema utilizado, muda a localização do logo no slide.
%\logo{\includegraphics[scale=0.08]{logo-utfpr.png}} 

\newtheorem{teo}{Teorema}[section]
\newtheorem{corolario}{Corolário}[section]
\newtheorem{lema}{Lema}[section]
%%%%%%%%%%%%%%%%%%%%%%%%%%%%%%%%%%%%%%%%%%%%%%


\begin{document}
%%%% Cada frame indica um slide %%%%

\begin{frame}
    \titlepage
\end{frame}

\begin{frame}{Sumário}
    \tableofcontents %Este comando é utilizado para inserir, automaticamente, um sumário com os títulos inseridos utilizando o comando section. Há outras maneiras de inserir títulos nos slides, como você verá a seguir, de modo que esses não apareçam no sumário.
\end{frame}

\section{Usando o comando section}
\begin{frame}{\insertsection}
    \begin{itemize}
        \item Podemos utilizar este comando para fazer o título do slide.
        \item Usado principalmente quando queremos fazer vários slides com o mesmo título.
        \item Dependendo do tema escolhido, poderá ser gerado um sumário automático (este vai aparecer em todos os slides), também com os títulos inseridos com o comando section (veja, por exemplo, com o tema Berkeley).
    \end{itemize}
\end{frame}

\begin{frame}{Título do slide} 
% Também dá para usar os comandos abaixo para fazer o título e o subtítulo
%\frametitle{O título pode ser inserido assim}
%\framesubtitle{Agora o subtítulo}
% Veja que quando o título é inserido de qualquer uma dessas maneiras, sem utilizar o section, ele não aparece no sumário.

    Aqui vem o texto do slide.

    \alert{Esse texto está em destaque/alerta.} 
    
    A cor do alerta pode mudar conforme o tema. 

\end{frame}

\begin{frame}{Inserindo blocos}  % Necessário escolher um tema para que realmente fique no formato de bloco (destacado). Caso contrário, aparece como texto normal.    
    \begin{block}{Título do bloco}
        Aqui vem um texto qualquer.
    \end{block}

    O estilo do bloco também muda de acordo com o tema.
    Veja abaixo o que acontece ao mudar o título de cada bloco.
    
    \begin{block}
        Aqui vem um texto qualquer.
    \end{block}

    \begin{block}{\ }
         Aqui vem um texto qualquer.
    \end{block}
\end{frame}

\begin{frame}{Outros tipos de bloco}
    \begin{exampleblock}{Exemplo}
        $$x^2-\frac{9}{25}=0$$
    \end{exampleblock}

% Se não centralizar a equação, fica assim:

    \begin{alertblock}{Alerta}
        $x=\pm\dfrac{3}{5}$
    \end{alertblock}

\end{frame}

\begin{frame}{Usando o comando pause}

    \textbf{Usado para separar o que vai aparecer no slide.} Veja o exemplo: \pause

    \begin{center}
        Essa parte aparece depois
    \end{center} \pause

    E essa por último.
    
\end{frame}

\section{Outros exemplos}
\begin{frame}{\insertsection}
% Para criar colunas no slide
    \begin{columns}[c]
        \begin{column}{5cm}
            Escreva aqui o texto que irá ficar na primeira coluna, que possui 5 centímetros de largura, conforme especificado acima. 
        \end{column}

        \begin{column}{5cm}
           Aqui deve ser o texto da segunda coluna, que também possui 5 centímetros de largura. Este tamanho pode ser alterado conforme a sua necessidade. Observe que cada coluna fica centralizada verticalmente na página quando utilizado o [c] no início do comando "columns''. Altere para [t] e veja o que acontece.
        \end{column}
    \end{columns}
% Para aumentar o número de colunas, é só inserir mais ambientes "column" dentro do ambiente "columns", especificando a largura de cada uma delas, respeitando os limites do slide.
\end{frame}

\begin{frame}{\insertsection}
    \begin{itemize}
        \item Para inserir listas, figuras e tabelas, são utilizados os mesmos comandos dos documentos em formato de artigo em \LaTeX. 
        \item Mais informações sobre a formatação do beamer no \href{http://www.telecom.uff.br/pet/petws/downloads/tutoriais/beamer/tut_beamer_2k100205.pdf}{link}.
    \end{itemize}
\end{frame}

\section{Ambiente teorema}
\begin{frame}{Exemplo do ambiente teorema}
    \begin{teo}[Teorema de Pitágoras]
    \label{theo:Pitagoras}
    Em todo triângulo retângulo, o quadrado do comprimento da hipotenusa é igual à soma dos quadrados dos comprimentos dos catetos. Esse resultado pode ser resumido usando-se a seguinte equação:
    \[c^2=a^2+b^2,\]
    em que $c$ representa o comprimento da hipotenusa, e $a$ e $b$ representam os comprimentos dos catetos.
    \end{teo}
\end{frame}


% Exemplo Corolário
% \begin{frame}{Exemplo Corolário}
%    Como consequência do Teorema de Pitágoras temos os seguintes corolários: 
%     \begin{corolario}
%     Em todo triângulo retângulo, o comprimento da hipotenusa é maior que o comprimento de qualquer um dos catetos e menor que a soma deles.
%     \end{corolario}

%     \begin{corolario}
%     Não existe triângulo retângulo cujos lados medem 1 cm, 2 cm e 3 cm.
%     \end{corolario}
% \end{frame}

% Exemplo lema 
% \begin{frame}{Exemplo Lema}
%     \begin{lema}
%     Seja $C$ um ponto em uma semi-reta definida pelos pontos $A$ e $B$, $S_{AB}$, se o segmento $AC$ é tal que $\overline{AC} < \overline{AB}$, então o ponto $C$ está entre $A$ e $B$.
%     \end{lema}
% \end{frame}



\end{document}